% Template from https://www.sharelatex.com/templates/journals/aps
%
% ****** Start of file apssamp.tex ******
%
%   This file is part of the APS files in the REVTeX 4.1 distribution.
%   Version 4.1r of REVTeX, August 2010
%
%   Copyright (c) 2009, 2010 The American Physical Society.
%
%   See the REVTeX 4 README file for restrictions and more information.
%
% TeX'ing this file requires that you have AMS-LaTeX 2.0 installed
% as well as the rest of the prerequisites for REVTeX 4.1
%
% See the REVTeX 4 README file
% It also requires running BibTeX. The commands are as follows:
%
%  1)  latex apssamp.tex
%  2)  bibtex apssamp
%  3)  latex apssamp.tex
%  4)  latex apssamp.tex
%
\documentclass[%
 reprint,
%superscriptaddress,
%groupedaddress,
%unsortedaddress,
%runinaddress,
%frontmatterverbose, 
%preprint,
%showpacs,preprintnumbers,
%nofootinbib,
%nobibnotes,
%bibnotes,
 amsmath,amssymb,
 aps,
%pra,
%prb,
%rmp,
%prstab,
%prstper,
%floatfix,
]{revtex4-1}


%%%%%%%%%%%%%%%%%%%%%%%%%%%%%%%%%%%%%%%%%%%%%%%%%%%%%%%
% dexy/pygments source code highlighting requirements %
\usepackage{xcolor}
\usepackage{fancyvrb}

% stylesheet for source code, generate using e.g. styles.sh
\usepackage{pastie} 

\usepackage{hyperref}
%%%%%%%%%%%%%%%%%%%%%%%%%%%%%%%%%%%%%%%%%%%%%%%%%%%%%%%


\usepackage{graphicx}% Include figure files
\usepackage{dcolumn}% Align table columns on decimal point
\usepackage{bm}% bold math
%\usepackage{hyperref}% add hypertext capabilities
%\usepackage[mathlines]{lineno}% Enable numbering of text and display math
%\linenumbers\relax % Commence numbering lines

%\usepackage[showframe,%Uncomment any one of the following lines to test 
%%scale=0.7, marginratio={1:1, 2:3}, ignoreall,% default settings
%%text={7in,10in},centering,
%%margin=1.5in,
%%total={6.5in,8.75in}, top=1.2in, left=0.9in, includefoot,
%%height=10in,a5paper,hmargin={3cm,0.8in},
%]{geometry}

\begin{document}

%\preprint{APS/123-QED}

\title{Dexy EinsteinPy Examples}% Force line breaks with \\
\thanks{EinsteinPy examples based upon EinsteinPy documentation (c) 2020 EinsteinPy Development Team, used under MIT license.}%

\date{\today}% It is always \today, today,
             %  but any date may be explicitly specified

\begin{abstract}
    Dexy makes it easy to write reproducible and recreatable documents in any
    document format, using any combination of programming languages. Dexy
    integrates prose, code, and data. 
\begin{description}
\item[Website]
https://dexy.it
\item[Package Source]
https://github.com/dexy/dexy
\item[Examples]
https://github.com/dexy/dexy-examples
\end{description}
\end{abstract}

\maketitle

%\tableofcontents

%%% "loading-json"
<% set snippets = d['snippets.json'].from_json() %>
%%% @end

\section{\label{sec:level1}Predefined Metrics in Symbolic Module Examples}

The \verb|einsteinpy.symbolic.prefedined| module contains many functions with predefined metrics.

\begin{description}
<% for k, v in snippets['predefined'].items() %>
    \item[<< k >>] << v['docs'] >>
<% endfor %>
\end{description}

\subsection{\label{sec:level2}Printing the metrics for visualization}

All the functions return instances of \verb|einsteinpy.symbolic.metric.MetricTensor|

Here are four examples:

\subsubsection{\label{sec:level3}Schwarzschild}
<< d['predefined_metrics.py|idio|l']['Schwarzschild'] >>
$<< snippets['Schwarzschild'] >>$

\subsubsection{\label{sec:level3}Minkowski}
<< d['predefined_metrics.py|idio|l']['Minkowski'] >>
$<< snippets['Minkowski'] >>$

\subsubsection{\label{sec:level3}DeSitter}
<< d['predefined_metrics.py|idio|l']['DeSitter'] >>
$<< snippets['DeSitter'] >>$

\subsubsection{\label{sec:level3}AntiDeSitter}
<< d['predefined_metrics.py|idio|l']['AntiDeSitter'] >>
$<< snippets['AntiDeSitter'] >>$

\subsection{\label{sec:level2}Calculating the scalar (Ricci) curavtures}


They should be constant for De-Sitter:
$<< snippets['DeSitter-curvature'] >>$

and Anti-De-Sitter spacetimes: \\
$<< snippets['AntiDeSitter-curvature'] >>$

Which simplifies to $<< snippets['AntiDeSitter-simplified'] >>$.


\section{\label{sec:level1}Using Dexy with EinsteinPy}

For a full introduction to \href{http://www.dexy.it/}{Dexy}, please look at the \href{http://www.dexy.it/docs/getting-started.html}{Getting Started Tutorial}.

Dexy runs by applying filters to files. Filters can transform file content in
arbitrary ways, including executing code and formatting text or data.

Here are some brief introductions to common filters we apply to source code files.

\subsection{\label{sec:level2}pyg \& idio}

The pyg filter applies \href{https://pygments.org/}{pygments syntax highlighting} to source code. At the end of this document you can see the complete \verb|predefined_metrics.py| file with syntax highlighting applied.

The idio filter uses the pygments system as well, but it first divides code up
into sections using specially formatted comments. This allows you present and
discuss code in manageable chunks, and also to leave out irrelevant parts
(perhaps include them in an appendix).

For example, here is just the custom pprint function definition:

<< d['predefined_metrics.py|idio|l']['custom-pprint'] >>


\subsection{\label{sec:level2}Running Code}

Next, we probably want to be able to actually run the code and see output it generates.

Depending on what we want to do with the output, we can take a variety of approaches.

\subsubsection{\label{sec:level3}Run With py Filter}

The py filter runs python code in the python compiler and saves just the output.

Here is what gets written to stdout by the \verb|predefined_metrics.py| script (long lines truncated for sanity):

\footnotesize
\begin{Verbatim}
<% for line in str(d['predefined_metrics.py|py']).splitlines() %>
<< line[0:60] >><% endfor %>
\end{Verbatim}

\subsubsection{\label{sec:level3}Run With pycon Filter}

An alternative, the pycon filter, runs Python more like a console or repl, and
shows you both the input and output. It's great for tutorials. It also respects
sections, so you can include just a section at a time.

For example, here is the pycon output of just the Minkowski section (syntax
highlighting is applied by subsequently passing it throug the pyg filter):

<< d['predefined_metrics.py|idio|pycon|pyg|l']['Minkowski'] >>

\subsubsection{\label{sec:level3}Generating Artifacts}

While including the results of stdout or a REPL-like view can be useful some of
the time, especially for code documentation/tutorials, for research papers or
work that's more about results, we probably want to have the code running
behind the scenes to generate artifacts which we can incorporate into documents
in a more flexible way. With Dexy, any data files you generate as side effects will be available to your final document, and you can include them flexibly.

For example, here is a \LaTeX\ rendering of the Schwarzschild tensor:

\vspace{5mm}

%%% "using-json"
$<< snippets['Schwarzschild'] >>$
%%% @end

\vspace{5mm}

This was generated by saving latex output to a dictionary, and then accessing the desired key from this document:

<< d['example.tex|idio|l']['loading-json'] >>
<< d['example.tex|idio|l']['using-json'] >>


\subsubsection{\label{sec:level3}Full Python Source Code}

\begin{widetext}
<< d['predefined_metrics.py|pyg|l'] >>
\end{widetext}

\end{document}
%
% ****** End of file apssamp.tex ******
